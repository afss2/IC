\documentclass[a4paper]{article}

%% Language and font encodings
\usepackage[brazil]{babel}
\usepackage[utf8x]{inputenc}
\usepackage[T1]{fontenc}
\fontsize{10}{10}\selectfont

%% Sets page size and margins
\usepackage[a4paper,top=3cm,bottom=2cm,left=3cm,right=3cm,marginparwidth=1.75cm]{geometry}

%% Useful packages
\usepackage{amsmath}
\usepackage{graphicx}
\usepackage[colorinlistoftodos]{todonotes}
\usepackage[colorlinks=true, allcolors=blue]{hyperref}
\usepackage{indentfirst}

\title{IF682 - Engenharia de Software e Sistemas}
\author{André Ferreira}

\begin{document}
\maketitle
\section{Introdução}
A disciplina de Engenharia de Software e Sistemas, dada no 4º período de Ciência da Computação tem como professor Paulo Borba, e como objetivo o ensinamento sobre criação de softwares e sistemas de qualidade, desenvolvimento do trabalho em equipe, gestão e desenvolvimento de projetos e a capacidade de comunicação. Abrange áreas de conhecimentos de software, como sua projeção, construção, manutenção, arquitetura etc, sendo como destaque os estudos de linguagem de programação, bando de dados e paradigmas de programação. É definida por Friedrich Ludwig Bauer como "Engenharia de Software é a criação e a utilização de sólidos princípios de engenharia a fim de obter software de maneira econômica, que seja confiável e que trabalhe em máquinas reais".

\section{Relevância}
 É uma cadeira muito importante, porque além de dar uma grande introdução a área de softwares e sistemas, treinar os alunos para projetos em grupos, acaba sendo um norte para quem deseja atuar nela e exige outras coisas dos alunos, como a ética, pontualidade, comprometimento, sendo as aulas para discussão do material lido \textbf{antes} da mesma, necessitando de uma dedicação aproximadamente de 12 horas semanais\cite{slides}.

É também uma das áreas tecnológicas mais importantes atualmente, citando Ian Sommerville: "Engenharia de Software é, portanto, uma tecnologia criticamente importante para o futuro da humanidade"\cite{Ian}. Sendo 

\end{document}